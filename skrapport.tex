%% chscite Skånings rapportklass
%%
%% Copyright (C) 2012-2013 by Simon Sigurdhsson <sigurdhsson@gmail.com>
%% 
%% This work may be distributed and/or modified under the
%% conditions of the LaTeX Project Public License, either version 1.3
%% of this license or (at your option) any later version.
%% The latest version of this license is in
%%   http://www.latex-project.org/lppl.txt
%% and version 1.3 or later is part of all distributions of LaTeX
%% version 2005/12/01 or later.
%% 
%% This work has the LPPL maintenance status `maintained'.
%% 
%% The Current Maintainer of this work is Simon Sigurdhsson.
%% 
%% This work consists of the file skrapport.tex and the 
%% derived files skrapport.cls, skrapport-colortheme-default.sty,
%% skrapport-colortheme-unscathed.sty, skrapport-colortheme-violet.sty
%% and skrapport-colortheme-cruelwater.sty.
\documentclass{skdoc}
\usepackage{hologo,booktabs}
\usepackage[style=authoryear]{biblatex}
\usepackage{csquotes}
%\usepackage{chslacite}

\ExplSyntaxOn
\cs_set_protected_nopar:Npn\ExplHack{
    \char_set_catcode_letter:n{ 58 }
    \char_set_catcode_letter:n{ 95 }
}
\ExplSyntaxOff

% Hide the implementation
%\OnlyDescription

% Bibliography entries
\begin{filecontents}{skrapport.bib}
    @article{kpfonts,
        author = {Christophe Caignaert},
        title = {KP-Fonts 3.31},
        year = {2010},
        url = {http://www.tex.ac.uk/tex-archive/fonts/kpfonts/doc/kpfonts.pdf}
    }
\end{filecontents}
\addbibresource{skrapport.bib}

% Declare the target files
\SelfPreambleTo{\mypreamble}
\DeclareFile[key=class,preamble=\mypreamble]{skrapport.cls}
\DeclareFile[key=theme-default,preamble=\mypreamble]%
    {skrapport-colortheme-default.sty}
\DeclareFile[key=theme-unscathed,preamble=\mypreamble]%
    {skrapport-colortheme-unscathed.sty}
\DeclareFile[key=theme-cruelwater,preamble=\mypreamble]%
    {skrapport-colortheme-cruelwater.sty}
\DeclareFile[key=theme-violet,preamble=\mypreamble]%
    {skrapport-colortheme-violet.sty}
\DeclareFile[key=theme-skdoc,preamble=\mypreamble]%
    {skrapport-colortheme-skdoc.sty}

% This is where the documentation begins
\begin{document}
    % Change & version info
    \version{0.11a}
    \changes{0.01}{Initial version}
    \changes{0.03}{Removed \cs{rd} and \cs{id}}
    \changes{0.04}{Added \pkg{microtype} package}
    \changes{0.05}{Improved documentation}
    \changes{0.06}{Corrected cheksum, further improved documentation}
    \changes{0.07}{Various bugfixes, \hologo{XeLaTeX} compatibility, 
                    better float settings, quote style fix,
                    \opt{intlimits} option to \pkg{amsmath}}
    \changes{0.07a}{Fixed \pkg{kpfonts} issues}
    \changes{0.09}{Introduced \pkg{kvoptions},
                    fixed abstract in twocolumn mode}
    \changes{0.10}{Include skmath if exists
                    and wanted. Gobble optional arguments to
                    \env{figure} and \env{table} in two-column mode.}
    \changes{0.10a}{Include \pkg{xparse} (fixes breakage).}
    \changes{0.11}{Added \opt{minion} for Adobe Minion Pro font}
    \changes{0.11a}{Added \opt{skdoc} font option and \thm{skdoc} color theme}

    % Metadata
    \package[ctan=skrapport,vcs=https://github.com/urdh/skrapport]{skrapport}
    \title{The \textbf{\thepackage} document class}
    \author{Simon Sigurdhsson}
    \email{sigurdhsson@gmail.com}

    % First page
    \maketitle
    \begin{abstract}
        A document class intended for simple documents \emph{e.g.}
        reports handed in to courses and such. It is small,
        straightforwars and heavily inspired by the Prac\TeX{}
        Journal style.
    \end{abstract}
    \tableofcontents

    \section{Documentation}
    \subsection{Loading the class}
    The document class is loaded using \cs{documentclass} as usual,
    but it has a bunch of options that you might want to know about.

    \subsubsection{Compatibility options}
    There are a couple of options that are mostly provided for
    compatibility with the standard \LaTeXe\ document classes. These
    will do exactly the same thing they do in the standard classes,
    although they may differ in default value. Although some of these
    are key-value options, for compatibility their valid values may
    also be passed as keys.

    \Options{paper}\WithValues{a4paper, a5paper}\AndDefault{a4paper}
    Specifying paper size is possible using the \opt{paper} option
    (the values of which are based
    on European paper sizes; use the \pkg{geometry} package for other
    standards). Only A4 and A5 are defined since these two sizes cover
    pretty much all intended use of the class. \opt{a4paper} is the
    default value of this option, and the only other valid value is
    \opt{a5paper}.

    \Options{ptsize}\WithValues{10pt, 11pt, 12pt}\AndDefault{11pt}
    The same font sizes specified in the standard classes are also
    available in \thepackage\ (\emph{i.e.} \opt{10pt},
    \opt{11pt} and \opt{12pt}. The default font size is
    \opt{11pt}, and there should be no compelling reason to change
    this.

    \Options{draft,final}
    The \opt{draft} and \opt{final} options work as expected, triggering
    or untriggering the familiar draft mode. The default is \opt{final}.

    \Options{fleqn,leqno}
    Purely for compatibility \thepackage\ also defines the \opt{fleqn}
    and \opt{leqno} options. As with the standard \LaTeXe\ classes,
    \opt{fleqn} aligns equations with the left-hand margin and
    \opt{leqno} places equation numbers to the left. None of these
    are activated by default.

    \subsubsection{Typographic options}
    There is also a number of options available to change certain
    aspects of the typography of the typeset document.

    \Options{titles}\WithValues{rm, bf, sf}\AndDefault{rm}
    There are three different ways to typeset section headings in
    \thepackage: \texttt{rm} (upright serif), \texttt{bf} (boldfaced
    serif) and \texttt{sf} (sans serif). The default is \texttt{rm}.

    \Options{font}\WithValues{nofont, lmodern, palatino, kpfonts, minion, skdoc}\AndDefault{kpfonts}
    In addition, the specific font can also be changed --- the available
    choices are \texttt{lmodern} (Latin Modern), \texttt{kpfonts}
    (Kp-Fonts), \texttt{minion} (Adobe Minion Pro), \texttt{palatino} (either \TeX-Gyre Pagella or
    Pazo Math depending on what's available) and \texttt{skdoc} (PT Serif and Open Sans, the style used by \pkg{skdoc}), with the default being
    \texttt{kpfonts}. It is also possible to tell \thepackage\ not
    to use any font (\texttt{nofont}), which is sometimes useful
    when using \hologo{XeLaTeX}, among other things.

    \Options{indent,noindent}
    Controlling indentation is posible using the options
    \texttt{(no)indent}. The default, \texttt{noindent}, behaves
    much like the \pkg{parskip} package in that it replaces
    paragraph indentation with vertical spacing.

    \Options{onecolumn,twocolumn}
    Similar to the options available in the standard classes, these
    options specify wether to typeset the document in one or two
    columns. Unlike the standard classes, the two-column mode is
    implemented using \pkg{multicol}. The default is \opt{onecolumn}.

    \subsubsection{Other options}
    \Options{swe,eng}
    Either \pkg{babel} or \pkg{polyglossia} (depending on engine) is
    loaded by the package. These options specify what language should
    be used as the main language (swedish or english); both languages
    are always loaded. The default is \opt{swe}.

    \Options{color,nocolor}
    It is also possible to load \pkg{xcolor} inside the package. If
    this is done, a range of color themes (discussed later) will be
    available in the package and these will affect the document. The
    default value is \opt{color}.

    \Options{math,nomath}
    Loading the \pkg{skmath} package is recommended and as such it is
    loaded by default. If this for some reason is undesirable, the
    \opt{nomath} option will supress this behaviour.

    \subsection{Macros and environments}
    In general, the class defines the same macros as the \pkg{article} class, and adds a few. Only the novel ones are described here, as
    the inherited ones should behave identically.

    \subsubsection{Front-matter and metadata}
    \DescribeMacro\license{<license>}
    The \cs{license} macro specifies the name of a license under which
    the document is available. This will be typeset on the lower right
    corner of the title page.

    \DescribeMacro\regarding{<text>}
    The class adds a \cs{regarding} macro, which is used like the 
    standard \cs{author} and \cs{title} macros and should be given an 
    accurate but short description of the purpose of the report (i.e.
    ~course name or similar). This is printed along with the date on 
    the top of the title/first page.

    \DescribeMacro\author[<email>]{<author>}
    The \cs{author} macro is redefined in two ways. To begin with, the
    macro now acceps an optional argument specifying the email address
    of the author. If the macro is used multiple times, authors are
    appended to the list of author names displayed by \cs{maketitle}.

    \DescribeMacro\maketitle
    The title page (or rather, block) has been refashioned to mimic the
    Prac\TeX\ Journal style. This means a fairly compact block, starting
    with a line of text containing the date and subject matter, followed
    by a large skip and then the title, author and optionally an
    abstract set ragged-right and fairly close together.

    \subsubsection{Useful macros}
    The class defines a few additional macros that aren't available in
    \pkg{article} but don't fit in any specifiv \enquote{set} of
    features. These include commands to typeset comments.

    \DescribeMacro\comment*{<comment>}
    \DescribeMacro\com*{<comment>}
    \DescribeMacro\note*{<comment>}
    The \cs{comment} macro (also available in an unstarred variant)
    typesets a comment. The starred variant typesets the commen in red
    prefixed by the word \enquote{Comment}, while the unstarred variant
    typesets the comment as a margin note (but still prefixed). The
    \cs{com} and \cs{note} macros are aliases of \cs{comment}.

    \subsubsection{Two-column mode}
    \DescribeEnv[<content>]{onecol}
    In \opt{twocolumn} mode, the package defines an environment
    \env{onecol} which typesets its contents in a single column. 
    Additionally, it redefines \env{figure} and \env{table} as
    non-floats, leaving the starred versions intact.

    \subsubsection{Color themes}
    \DescribeMacro\colortheme{<theme>}
    If the package is loaded with the \opt{color} option, changing the color theme is
    possible using \cs{colortheme}, which loads an
    appropriate package. At the moment, four color themes are available.

    \Theme{default}
    The \thm{default} theme is fairly conservative, only coloring
    \pkg{hyperef} links with more readable, slightly darker colors than
    the standard ones. It should print well even on non-color printers.

    \Theme{unscathed}
    The \thm{unscathed} theme is based on a palette with the same
    name on COLOURlovers%
\footnote{\url{http://www.colourlovers.com/palette/1440498/unscathed}},
    and applies a \textcolor[HTML]{463335}{dark brown} color to
    emphasized text, a \textcolor[HTML]{CF5D3B}{rusty} color to links,
    a \textcolor[HTML]{B34430}{darker rust} color to titles and a
    \textcolor[HTML]{70524A}{lighter brown} to quotes.

    \Theme{cruelwater}
    The \thm{cruelwater} theme is also based on a palette from 
    COLOURlovers%
\footnote{\url{http://www.colourlovers.com/palette/126030/Cruel_Water_at_Night}},
    and applies a \textcolor[HTML]{030C22}{dark blue} color to bold
    text and captions, a \textcolor[HTML]{20293F}{slightly less dark 
    blue} to titles and emphasized text, a \textcolor[HTML]{A9B0B3}{
    light gray} color to small print and a \textcolor[HTML]{404749}{
    darker gray} to quotes.

    \Theme{violet}
    The \thm{violet} theme, like \thm{unscathed} and \thm{cruelwater},
    is based on a COLOURlovers palette%
\footnote{\url{http://www.colourlovers.com/palette/1831303/Violet_White_Bedrm}}.
    It colors all links \textcolor[HTML]{932444}{bright purple}, applies
    a \textcolor[HTML]{311A2A}{dark puple} color to titles, bold text
    and captions, a \textcolor[HTML]{D6CBCF}{grayish purple} to small
    print, a \textcolor[HTML]{463335}{dark brown} color to quotes and a
    \textcolor[HTML]{98758D}{pastel violet} color to emphasized text.

    \Theme{skdoc}
    The \thm{skdoc} theme is loosely based on the \pkg{skdoc} document class, with which this documentation is typeset.

    \subsection{Additional information}
    The document class includes a number of packages by default. This
    is useful to know, since passing explicit options to these packages
    will require you to utilize the \cs{PassOptionsToPackage} macro
    before you load the class using \cs{documentclass}.
    Table~\ref{tab:pkgs} lists the packages included by \thepkg\ along
    with their options (if applicable).

    \begin{table}[tbp]
        \centering
        \caption{User-level packages included by \thepkg.}
        \label{tab:pkgs}
        \begin{tabular}{llp{15em}}
            \toprule
            \textbf{Package} & \textbf{Options} & \textbf{Comments} \\
            \midrule
            \pkg{amsmath} & \texttt{intlimits} & Provides \hologo{AmS} commands and environments. \\
            \pkg{amssymb} & & Only if not using \opt{kpfonts}.\\
            \pkg{babel} & see options \opt{swe} and \opt{eng} & Only loaded if \emph{not} using \hologo{XeTeX}.\\
            \pkg{calc} & & \\
            \pkg{fontenc} & \texttt{T1} & Only loaded if \emph{not} using \hologo{XeTeX}. Makes sure we are using a good font encoding for crisp appearance on-screen (OT1 is horrible). \\
            \pkg{fontspec} & \texttt{quiet} & Only loaded if using \hologo{XeTeX}. Provides basic OTF font selection commands.\\
            \pkg{geometry} & \texttt{a4paper} or \texttt{a5paper} & This is used by the \opt{paper} option to set the paper area. \\
            \pkg{icomma} & & \\
            \pkg{inconsolata} & & \\
            \pkg{microtype} & & Provides micro-typographic improvements.\\
            \pkg{multicol} & & Only loaded with the \opt{twocolumn} option. \\
            \pkg{polyglossia} & see options \opt{swe} and \opt{eng} & Only loaded if using \hologo{XeTeX}.\\
            \pkg{skmath} & & Only if it exists and \opt{nomath} isn't set.\\
            \pkg{textcomp} & & Only if not using \opt{kpfonts}.\\
            \pkg{xcolor} & & Only loaded with the \opt{color} option.\\
            \bottomrule
        \end{tabular}
    \end{table}

    \section{Known issues}
    A list of current issues is available in the Github repository of this
    package\footnote{\url{https://github.com/urdh/skrapport/issues}}, but as
    of the release of \theversion, there are no known issues:
    %\begin{description}
    %    \item[\#6]  ???
    %\end{description}
    
    If you discover any bugs in this package, please report them to the issue
    tracker in the \thepackage\ Github repository.

    \Implementation\ExplHack
    \section{Implementation}
    Start with the standard \LaTeX\ cruft.
\begin{MacroCode}{class}
\NeedsTeXFormat{LaTeX2e}[1995/12/01]
\ProvidesClass{skrapport}%
    [2013/04/10 v0.11a Skånings rapportklass]
\end{MacroCode}
    Then, start by including \pkg{l3keys2e} and defining some
    variables for future use.
\begin{MacroCode}{class}
\RequirePackage{expl3,l3keys2e,xparse,xstring}
\ExplSyntaxOn
\bool_new:N\g__skrapport_draft_bool
\bool_new:N\g__skrapport_color_bool
\bool_new:N\g__skrapport_has_polyglossia_bool
\end{MacroCode}
    
    \subsection{Messages}
\begin{MacroCode}{class}
\msg_new:nnnn{skrapport}{option-deprecated}{Option~`#1~deprecated!}
    {Please~use~`#2'~instead.}
\msg_new:nnnn{skrapport}{option-no-effect}{Option~`#1'~deprecated!}
    {It~has~no~effect;~simply~use~nothing~instead.}
\msg_new:nnnn{skrapport}{invalid-paper-size}{Invalid~paper~size~`#1'!}
    {\token_to_str:N\__skrapport_setup_paper:n~was~invoked~with~an~
     invalid~argument;~paper~size~will~remain~unchanged.}
\msg_new:nnnn{skrapport}{invalid-point-size}{Invalid~point~size~`#1'!}
    {\token_to_str:N\__skrapport_setup_ptsize:n~was~invoked~with~an~
     invalid~argument;~cannot~continue~without~setting~a~valid~point~
     size.~Please~fix~the~issue~before~typesetting~again.}
\msg_new:nnnn{skrapport}{invalid-titles}{Invalid~title~type~`#1'!}
    {\token_to_str:N\__skrapport_setup_titles:n~was~invoked~with~an~
     invalid~argument;~falling~back~to~`#2'.}
\msg_new:nnnn{skrapport}{invalid-lang}{Invalid~language~`#1'!}
    {\token_to_str:N\__skrapport_setup_lang:n~was~invoked~with~an~
     invalid~argument;~`babel'~and/or~`polyglossia'~will~remain~
    unloaded.}
\msg_new:nnnn{skrapport}{invalid-font}{Invalid~font~`#1'!}
    {\token_to_str:N\__skrapport_setup_font:n~was~invoked~with~an~
     invalid~argument;~no~font~package~has~been~loaded.~This~may~result~
     in~bad~rendering~due~to~old~Computer~Modern~fonts.}
\msg_new:nnnn{skrapport}{noop-call}{No-op~invokation~of~macro!}
    {\token_to_str:N#1~was~invoked,~but~circumstances~dictate~that~the~
    macro~should~do~nothing.~Will~do~\token_to_str:N\prg_do_nothing:~
    instead.}

\msg_new:nnn{skrapport}{not-implemented}
    {Unimplemented~macro~\token_to_str:N#1!}
\end{MacroCode}

    \subsection{Options}
\begin{MacroCode}{class}
\keys_define:nn{skrapport}{
\end{MacroCode}
    \begin{option}{paper}{a4paper, a5paper, a4, a5}
    \begin{option}{a4paper}
    \begin{option}{a5paper}
    Declare the paper size options.
\begin{MacroCode}{class}
    paper .choice:,
    paper / a4 .code:n =
        {\__skrapport_setup_paper:n{a4}},
    paper / a5 .code:n =
        {\__skrapport_setup_paper:n{a5}},
    paper / a4paper .code:n = 
        {\msg_warning:nnnn{skrapport}{option-deprecated}
            {paper=a4paper}{paper=a4}
         \keys_set:nn{skrapport}{paper=a4}},
    paper / a5paper .code:n = 
        {\msg_warning:nnnn{skrapport}{option-deprecated}
            {paper=a5paper}{paper=a5}
         \keys_set:nn{skrapport}{paper=a4}},
    paper .value_required:,
    paper .initial:n = a4,
    a4paper .code:n = 
        {\msg_warning:nnnn{skrapport}{option-deprecated}
            {a4paper}{paper=a4}
         \keys_set:nn{skrapport}{paper=a4}},
    a5paper .code:n = 
        {\msg_warning:nnnn{skrapport}{option-deprecated}
            {a5paper}{paper=a5}
         \keys_set:nn{skrapport}{paper=a5}},
\end{MacroCode}
    \end{option}
    \end{option}
    \end{option}

    \begin{option}{ptsize}{10pt, 11pt, 12pt}
    \begin{option}{10pt}
    \begin{option}{11pt}
    \begin{option}{12pt}
    Declare point size options.
\begin{MacroCode}{class}
    ptsize .choices:nn =
        { 10pt, 11pt, 12pt }
        {\__skrapport_setup_ptsize:x{\tl_use:N\l_keys_choice_tl}},
    ptsize .value_required:,
    ptsize .initial:n = 11pt,
    10pt .meta:n = {ptsize=10pt},
    11pt .meta:n = {ptsize=11pt},
    12pt .meta:n = {ptsize=12pt},
\end{MacroCode}
    \end{option}
    \end{option}
    \end{option}
    \end{option}

    \begin{option}{twocolumn}
    \changes{0.08}{Added option \opt{twocolumn}}
    \begin{option}{onecolumn}
    Declare column options.
\begin{MacroCode}{class}
    twocolumn .code:n =
        {\__skrapport_setup_twocolumn:},
    onecolumn .code:n =
        {\msg_warning:nnn{skrapport}{option-no-effect}{onecolumn}},
\end{MacroCode}
    \end{option}
    \end{option}

    \begin{option}{draft}{true, false}
    \begin{option}{final}
    Declare \opt{draft} and \opt{final} options.
\begin{MacroCode}{class}
    draft .choice:,
    draft / true .code:n =
        {\bool_gset_true:N\g__skrapport_draft_bool
         \__skrapport_setup_draft:},
    draft / false .code:n = 
        {\bool_gset_false:N\g__skrapport_draft_bool},
    draft .default:n = true,
    draft .initial:n = false,
    final .code:n =
        {\msg_warning:nnn{skrapport}{option-no-effect}{final}},
\end{MacroCode}
    \end{option}
    \end{option}


    Declare the \opt{fleqn} and \opt{leqno} options for
    compatibility with the \pkg{article} class.
    \begin{option}{leqno}
\begin{MacroCode}{class}
    leqno .code:n =
        {\__skrapport_setup_leqno:},
\end{MacroCode}
    \end{option}
    \begin{option}{fleqn}
\begin{MacroCode}{class}
    fleqn .code:n =
        {\__skrapport_setup_fleqn:},
\end{MacroCode}
    \end{option}

    \begin{option}{titles}{rm, bf, sf}
    Declare options for section titles.
    \begin{option}{rmtitles}
    \begin{option}{bftitles}
    \begin{option}{sftitles}
    \changes{0.09}{Added \opt{sftitles} option}
\begin{MacroCode}{class}
    titles .choices:nn =
        { rm, bf, sf }
        {\__skrapport_setup_titles:x{\tl_use:N\l_keys_choice_tl}},
    titles .value_required,
    titles .initial:n = bf,
    rmtitles .code:n = 
        {\msg_warning:nnnn{skrapport}{option-deprecated}
            {rmtitles}{titles=rm}
         \keys_set:nn{skrapport}{titles=rm}},
    bftitles .code:n = 
        {\msg_warning:nnnn{skrapport}{option-deprecated}
            {bftitles}{titles=bf}
         \keys_set:nn{skrapport}{titles=bf}},
    sftitles .code:n = 
        {\msg_warning:nnnn{skrapport}{option-deprecated}
            {sftitles}{titles=sf}
         \keys_set:nn{skrapport}{titles=sf}},
\end{MacroCode}
    \end{option}
    \end{option}
    \end{option}
    \end{option}

    \begin{option}{lang}{sv, en}
    \begin{option}{swe}
    \begin{option}{eng}
    Declare options for swedish/english \pkg{babel} or
    \pkg{polyglossia} support.
\begin{MacroCode}{class}
    lang .choices:nn = 
        { sv, en }
        {\__skrapport_setup_lang:x{\tl_use:N\l_keys_choice_tl}},
    lang .value_required:,
    lang .initial:n = sv,
    swe .code:n = 
        {\msg_warning:nnnn{skrapport}{option-deprecated}
            {swe}{lang=sv}
         \keys_set:nn{skrapport}{lang=sv}},
    eng .code:n = 
        {\msg_warning:nnnn{skrapport}{option-deprecated}
            {eng}{lang=en}
         \keys_set:nn{skrapport}{lang=en}},
\end{MacroCode}
    \end{option}
    \end{option}
    \end{option}

    \begin{option}{font}{none, kpfonts, lmodern, palatino, minion, skdoc}
    Declare font options.
    \begin{option}{kpfonts}
    \changes{0.07}{Added option \opt{kpfonts}, set as default
                    option for fonts}
    \begin{option}{lmodern}
    \begin{option}{palatino}
    \begin{option}{minion}
    \changes{0.11}{Added option \opt{minion}}
    \begin{option}{skdoc}
    \changes{0.11a}{Added option \opt{skdoc}}
    \begin{option}{nofont}
\begin{MacroCode}{class}
    font .choices:nn =
        { none, kpfonts, lmodern, palatino, minion, skdoc }
        {\__skrapport_setup_font:x{\tl_use:N\l_keys_choice_tl}},
    font .value_required:,
    font .initial:n = kpfonts,
    nofont .code:n = 
        {\msg_warning:nnnn{skrapport}{option-deprecated}
            {nofont}{font=none}
         \keys_set:nn{skrapport}{font=none}},
    kpfonts .code:n = 
        {\msg_warning:nnnn{skrapport}{option-deprecated}
            {kpfonts}{font=kpfonts}
         \keys_set:nn{skrapport}{font=kpfonts}},
    lmodern .code:n = 
        {\msg_warning:nnnn{skrapport}{option-deprecated}
            {lmodern}{font=lmodern}
         \keys_set:nn{skrapport}{font=lmodern}},
    palatino .code:n = 
        {\msg_warning:nnnn{skrapport}{option-deprecated}
            {palatino}{font=palatino}
         \keys_set:nn{skrapport}{font=palatino}},
    minion .code:n = 
        {\msg_warning:nnnn{skrapport}{option-deprecated}
            {minion}{font=minion}
         \keys_set:nn{skrapport}{font=minion}},
    skdoc .code:n = 
        {\msg_warning:nnnn{skrapport}{option-deprecated}
            {skdoc}{font=skdoc}
         \keys_set:nn{skrapport}{font=skdoc}},
\end{MacroCode}
    \end{option}
    \end{option}
    \end{option}
    \end{option}
    \end{option}
    \end{option}
    \end{option}

    \begin{option}{indent}{true, false}
    \changes{0.02}{Added option of indented paragraphs}
    \begin{option}{noindent}
    Declare indentation options.
\begin{MacroCode}{class}
    indent .choice:,
    indent / true .code:n = {\prg_do_nothing:},
    indent / false .code:n = {\__skrapport_setup_parskip:},
    indent .default:n = true,
    indent .initial:n = false,
    noindent .code:n =
        {\msg_warning:nnn{skrapport}{option-no-effect}{noindent}},
\end{MacroCode}
    \end{option}
    \end{option}

    \begin{option}{color}{true, false}
    \changes{0.09}{Added \opt{color} option}
    \begin{option}{nocolor}
    Declare color options.
\begin{MacroCode}{class}
    color .choice:,
    color / true .code:n =
        {\bool_gset_true:N\g__skrapport_color_bool
         \__skrapport_setup_color:},
    color / false .code:n = 
        {\bool_gset_false:N\g__skrapport_color_bool},
    color .default:n = true,
    color .initial:n = true,
    nocolor .meta:n = {color=false},
\end{MacroCode}
    \end{option}
    \end{option}

    \begin{option}{math}{true, false}
    \begin{option}{nomath}
    Declare math options.
\begin{MacroCode}{class}
    nomath .choice:,
    nomath / false = .code:n {\__skrapport_setup_math:},
    nomath / true = .code:n {\prg_do_nothing},
    nomath .default:n = true,
    nomath .initial:n = false,
    math .code:n =
        {\msg_warning:nnn{skrapport}{option-no-effect}{math}},
\end{MacroCode}
    \end{option}
    \end{option}
\begin{MacroCode}{class}
}
\ExplSyntaxOff
\end{MacroCode}

    \section{Option setup macros}
    The ordering of these things have to be redone I think.

    \begin{macro}{\__skrapport_setup_paper:n}[1]
        {Paper size (\texttt{a4} or \texttt{a5})}
\begin{MacroCode}{class}
\cs_new:Npn\__skrapport_setup_paper:n#1{
    \str_case:nnn{#1}{
        {a4}{
            \RequirePackage[a4paper]{geometry}
            \setlength\paperheight {297mm}
            \setlength\paperwidth  {210mm}
        }
        {a5}{
            \RequirePackage[a5paper]{geometry}
            \setlength\paperheight {210mm}
            \setlength\paperwidth  {148mm}
        }
    }{
        \msg_warning:nnn{skrapport}{invalid-paper-size}{#1}
    }
}
\end{MacroCode}
    \end{macro}

    \begin{macro}{\__skrapport_setup_ptsize:n}[1]
        {Point size (\texttt{10pt}, \texttt{11pt} or \texttt{12pt})}
\begin{MacroCode}{class}
\cs_new:Npn\__skrapport_setup_ptsize:n#1{
    \str_case:nnn{#1}{
\end{MacroCode}
    \begin{macro*}{\c__skrapport_ptsize_token}
\begin{MacroCode}{class}
        {10pt}{\token_new:Nn\c__skrapport_ptsize_token{0}}
        {11pt}{\token_new:Nn\c__skrapport_ptsize_token{1}}
        {12pt}{\token_new:Nn\c__skrapport_ptsize_token{2}}
\end{MacroCode}
    \end{macro*}
\begin{MacroCode}{class}
    }{
        \msg_critical:nnn{skrapport}{invalid-point-size}{#1}
    }
    \file_input:n{size1\c__skrapport_ptsize_token .clo}
}
\end{MacroCode}
    \end{macro}

    \begin{macro}{\__skrapport_setup_twocolumn:}
\begin{MacroCode}{class}
\cs_new:Nn\__skrapport_setup_twocolumn:{
    \RequirePackage{etoolbox}
    \RequirePackage{multicol}
    \msg_fatal:nnn{skrapport}{not-implemented}
        {\AtEndClass{\__skrapport_setup_twocolumn:}}
}
\end{MacroCode}
    \end{macro}

    \begin{macro}{\__skrapport_setup_leqno:}
\begin{MacroCode}{class}
\cs_new:Nn\__skrapport_setup_leqno:{
    \RenewDocumentCommand\@eqnnum{}{
        \hbox_to_wd:nn{.01pt}{}
        \hbox_overlap_right:n{
            \normalfont\normalcolor
            \skip_horizontal:n{-\displaywidth}
            (\theequation)
        }
    }
}
\end{MacroCode}
    \end{macro}

    \begin{macro}{\__skrapport_setup_fleqn:}
\begin{MacroCode}{class}
\cs_new:Nn\__skrapport_setup_fleqn:{
    \dim_new:N\mathindent
    \AtEndOfClass{\dim_gset_eq:NN\mathindent\leftmargini}
    \RenewDocumentCommand\[{}{
        \ifmmode\@badmath\else
            \begin{trivlist}
                \@beginparpenalty\predisplaypenalty
                \@endparpenalty\postdisplaypenalty
                \item[]\leavevmode
                \hbox_to_wd:nn{\linewidth}\group_begin:
                \c_math_toggle_token\m@th\displaystyle
                \skip_horizontal:n{\mathindent}\group_begin:
        \fi
    }
    \RenewDocumentCommand\]{}{
        \ifmmode
                \group_end:
                \c_math_toggle_token\hfil
                \group_end:
            \end{trivlist}
        \else\@badmath\fi
    }
    \RenewDocumentEnvironment{equation}{}{
        \@beginparpenalty\predisplaypenalty
        \@endparpenalty\postdisplaypenalty
        \refstepcounter{equation}
        \trivlist
            \item[]\leavevmode
            \hbox_to_wd:nn{\linewidth}\group_begin:
            \c_math_toggle_token\m@th\displaystyle
            \skip_horizontal:n{\mathindent}
    }{
            \c_math_toggle_token\hfil
            \dim_set_eq:NN\displaywidth\linewidth
            \hbox:n{\@eqnnum}
            \group_end:
        \endtrivlist
    }
    \RenewDocumentEnvironment{eqnarray}{}{
        \stepcounter{equation}
        \cs_new:Nn\@currentlabel{\p@equation\theequation}
        \global\@eqnswtrue\m@th % !!!
        \dim_gzero:N\@eqcnt
        \skip_set_eq:NN\tabskip\mathindent
        \cs_set_eq:NN\\\@eqncr
        \skip_set_eq:NN\abovedisplayskip\topsep
        \ifvmode
            \skip_add:Nn\abovedisplayskip{\partopsep}
        \fi
        \skip_add:Nn\abovedisplayskip{\parskip}
        \skip_set_eq:NN\belowdisplayskip\abovedisplayskip
        \skip_set_eq:NN\belowdisplayshortskip\abovedisplayskip
        \skip_set_eq:NN\abovedisplayshortskip\abovedisplayskip
        \c_math_toggle_token\c_math_toggle_token
        \everycr{}\halign to\linewidth % !!!
        \group_begin:
            \skip_horizontal:n{\@centering}
            \c_math_toggle_token\displaystyle
            \skip_zero:N\tabskip
            {##}\c_math_toggle_token
            \eqnsel&
            \global\@eqcnt\@ne % !!!
            \skip_horizontal:n{2\arraycolsep}
            \hfil\c_math_toggle_token
            {##}\c_math_toggle_token
            \hfil&
            \global\@eqcnt\tw@ % !!!
            \skip_horizontal:n{2\arraycolsep}
            \c_math_toggle_token\displaystyle
            {##}\c_math_toggle_token\hfil
            \skip_set_eq:NN\tabskip\@centering
            \global\@eqcnt\thr@@ % !!!
            \hbox_to_wd:nn{\c_zero_dim}\group_begin:
                \hss##
            \group_end:
            \skip_zero:N\tabskip
            \cr
    }{
            \@@eqncr
        \group_end:
        \global\advance\c@equation\m@ne % !!!
        \c_math_toggle_token\c_math_toggle_token
        \@ignoretrue
    }
}
\end{MacroCode}
    \end{macro}

    \begin{macro}{\__skrapport_setup_titles:n}[1]
        {Title style (\texttt{rm}, \texttt{bf} or \texttt{sf})}
\begin{MacroCode}{class}
\cs_new:Npn\__skrapport_setup_titles:n#1{
    \str_case:nnn{#1}{
\end{MacroCode}
    \begin{macro*}{\__skrapport_title_style:}
        \changes{0.11a}{Removed incorrect \cs{bfseries}, replaced
                        \cs{relax} with \cs{rmfamily}}
\begin{MacroCode}{class}
        {rm}{\cs_new:Nn\__skrapport_title_style:{\rmfamily}}
        {bf}{\cs_new:Nn\__skrapport_title_style:{\bfseries}}
        {sf}{\cs_new:Nn\__skrapport_title_style:{\sffamily}}
\end{MacroCode}
    \end{macro*}
\begin{MacroCode}{class}
    }{
        \msg_info:nnnn{skrapport}{invalid-titles}{#1}{rm}
        \__skrapport_setup_titles:n{rm}
    }
}
\end{MacroCode}
    \end{macro}

    \begin{macro}{\__skrapport_setup_lang:n}[1]
        {Language (\texttt{sv} or \texttt{en})}
\begin{MacroCode}{class}
\cs_new:Npn\__skrapport_setup_lang:n#1{
    \str_case:nnn{#1}{
        {sv}{
            \bool_if:NTF\g__skrapport_has_polyglossia_bool{
                \setmainlanguage{swedish}
                \setotherlanguage[variant=british]{english}
            }{
                \RequirePackage[british,swedish]{babel}
            }
        }
        {en}{
            \bool_if:NTF\g__skrapport_has_polyglossia_bool{
                \setmainlanguage[variant=british]{english}
                \setotherlanguage{swedish}
            }{
                \RequirePackage[swedish,british]{babel}
            }
        }
    }{
        \msg_info:nnn{skrapport}{invalid-lang}{#1}
    }
}
\end{MacroCode}
    \end{macro}

    \begin{macro}{\__skrapport_setup_font:n}[1]
        {Font (\texttt{none}, \texttt{kpfonts}, \texttt{lmodern},
         \texttt{palatino}, \texttt{minion} or \texttt{skdoc})}
\begin{MacroCode}{class}
\cs_new:Npn\__skrapport_setup_font:n#1{
    \RequirePackage[scaled=1.03]{inconsolata}
    \str_case:nnn{#1}{
        {none}{\prg_do_nothing:}
        {kpfonts}{
            \RequirePackage[easyscsl,intlimits,sumlimits]{kpfonts}
        }
        {lmodern}{
            \RequirePackage{lmodern}
        }
        {palatino}{
            \file_if_exist:nTF{tgpagella.sty}{
                \RequirePackage{tgpagella}
            }{
                \RequirePackage[osf]{mathpazo}
            }
        }
        {minion}{
            \RequirePackage{MinionPro}
            \RequirePackage{MnSymbol}
        }
        {skdoc}{
            \RequirePackage{PTSerif}
            \RequirePackage[defaultsans,osfigures,scale=0.95]{opensans}
        }
    }{
        \msg_info:nnn{skrapport}{invalid-font}{#1}
    }
    \str_if_eq:nF{#1}{kpfonts}{
        \RequirePackage[intlimits]{amsmath}
        \RequirePackage{amssymb}
        \RequirePackage{textcomp}
    }
}
\end{MacroCode}
    \end{macro}

    \begin{macro}{\__skrapport_setup_parskip:}
    This part mixes \LaTeXe and \LaTeX3, which may not be a good thing.
    On the other hand, it seems hard to avoid.
\begin{MacroCode}{class}
\cs_new:Nn\__skrapport_setup_parskip:{
    \skip_gset:Nn\parskip{0.5\baselineskip plus 2pt}
    \dim_gzero:N\parindent
    \skip_gset:Nn\parfillskip{30pt plus 1fil}
\end{MacroCode}
    \begin{macro*}{\@listI}
    \begin{macro*}{\@listi}
\begin{MacroCode}{class}
    \cs_gset:Npn\@listI{
        \dim_gset_eq:NN\leftmargin\leftmargini
        \dim_gset_eq:NN\parsep\parskip
        \dim_gzero:N\topsep
        \dim_gzero:N\itemsep
    }
    \cs_gset_eq:NN\@listi\@listI
\end{MacroCode}
    \end{macro*}
    \end{macro*}
\begin{MacroCode}{class}
    \@listi
\end{MacroCode}
    \begin{macro*}{\@listii}
\begin{MacroCode}{class}
    \cs_gset:Npn\@listii{
        \dim_gset_eq:NN\leftmargin\leftmarginii
        \dim_gset_eq:NN\labelwidth\leftmarginii
        \dim_gset_eq:NN\parsep\parskip
        \dim_gsub:Nn\labelwidth{-\labelsep}
        \dim_gzero:N\topsep
        \dim_gzero:N\itemsep
    }
\end{MacroCode}
    \end{macro*}
    \begin{macro*}{\@listiii}
\begin{MacroCode}{class}
    \cs_gset:Npn\@listiii{
        \dim_gset_eq:NN\leftmargin\leftmarginiii
        \dim_gset_eq:NN\labelwidth\leftmarginiii
        \dim_gset_eq:NN\parsep\parskip
        \dim_gsub:Nn\labelwidth{-\labelsep}
        \dim_gzero:N\topsep
        \dim_gzero:N\itemsep
    }
\end{MacroCode}
    \end{macro*}
\begin{MacroCode}{class}
    \dim_gzero:N\partopsep
\end{MacroCode}
    \begin{macro*}{\@starttoc}
\begin{MacroCode}{class}
    \RenewDocumentCommand\@starttoc{m}{
        \group_begin:
        \dim_gzero:N\parskip
        \file_if_exist:nT{\c_job_name_tl.#1}{
            \file_input:n{\c_job_name_tl.#1}
        }
        \if@filesw
            \iow_new:c{tf@#1}
            \iow_open:cn{tf@#1}{\c_job_name_tl.#1}
        \fi
        \@nobreakfalse
        \group_end:
    }
\end{MacroCode}
    \end{macro*}
\begin{MacroCode}{class}
}
\end{MacroCode}
    \end{macro}

    \begin{macro}{\__skrapport_setup_math:}
\begin{MacroCode}{class}
\cs_new:Nn\__skrapport_setup_math:{
    \file_if_exist:nT{skmath.sty}{\RequirePackage{skmath}}
}
\end{MacroCode}
    \end{macro}

    \begin{macro}{\__skrapport_setup_draft:}
\begin{MacroCode}{class}
\cs_new:Nn\__skrapport_setup_draft:{
    \bool_if:NTF\g__skrapport_draft_bool{
        \dim_gset:Nn\overfullrule{5pt}
    }{
        \msg_log:nnn{skrapport}{noop-call}{\__skrapport_setup_draft:}
        \prg_do_nothing:
    }
}
\end{MacroCode}
    \end{macro}

    \begin{macro}{\__skrapport_setup_color:}
\begin{MacroCode}{class}
\cs_new:Nn\__skrapport_setup_color:{
    \bool_if:NTF\g__skrapport_color_bool{
        \RequirePackage{etoolbox}
        \RequirePackage{xcolor}
    }{
        \msg_log:nnn{skrapport}{noop-call}{\__skrapport_setup_color:}
        \prg_do_nothing:
    }
}
\end{MacroCode}
    \end{macro}

    \begin{macro}{\__skrapport_setup_xelatex:}
\begin{MacroCode}{class}
\cs_new:Nn\__skrapport_setup_xelatex:{
    \RequirePackage[quiet]{fontspec}
\end{MacroCode}
    \begin{macro}{\nobreakspace}
\begin{MacroCode}{class}
    \DeclareDocumentCommand\nobreakspace{}{\leavevmode\nobreak\space}
\end{MacroCode}
    \end{macro}
\begin{MacroCode}{class}
    \RequirePackage{polyglossia}
    \bool_gset_true:N\g__skrapport_has_polyglossia_bool
}
\end{MacroCode}
    \end{macro}

    \begin{macro}{\__skrapport_setup_pdftex:}
\begin{MacroCode}{class}
\cs_new:Nn\__skrapport_setup_pdftex:{
    \RequirePackage[T1]{fontenc}
    \bool_gset_false:N\g__skrapport_has_polyglossia_bool
}
\end{MacroCode}
    \end{macro}

    \begin{macro}{\__skrapport_setup_engine_specific:}
\begin{MacroCode}{class}
\cs_new:Nn\__skrapport_setup_engine_specific:{
    \bool_if:nTF{\luatex_if_engine_p: || \xetex_if_engine_p:}{
        \__skrapport_setup_xelatex:
    }{
        \__skrapport_setup_pdftex:
    }
}
\end{MacroCode}
    \end{macro}

    \subsubsection{Variants with expandable arguments}
    \begin{macro}{\__skrapport_setup_ptsize:x}
\begin{MacroCode}{class}
\cs_generate_variant:Nn\__skrapport_setup_ptsize:n{ x }
\end{MacroCode}
    \end{macro}
    \begin{macro}{\__skrapport_setup_titles:x}
\begin{MacroCode}{class}
\cs_generate_variant:Nn\__skrapport_setup_titles:n{ x }
\end{MacroCode}
    \end{macro}
    \begin{macro}{\__skrapport_setup_lang:x}
\begin{MacroCode}{class}
\cs_generate_variant:Nn\__skrapport_setup_lang:n{ x }
\end{MacroCode}
    \end{macro}
    \begin{macro}{\__skrapport_setup_font:x}
\begin{MacroCode}{class}
\cs_generate_variant:Nn\__skrapport_setup_font:n{ x }
\end{MacroCode}
    \end{macro}

    Execute default options and process given options.
\begin{MacroCode}{class}
\__skrapport_setup_engine_specific:
\ProcessKeysOptions{skrapport}
\end{MacroCode}

    \subsection{Loading packages}
    \subsubsection{Optional packages}
    Require packages as per given options. Start with the paper size.
\begin{MacroCode}{class}

\end{MacroCode}

    \begin{macro}{\@ptsize}
    Follow up with the point size.
\begin{MacroCode}{class}

\end{MacroCode}
    \end{macro}

    Setting \opt{draft} mode is fairly easy, we only need
    \cs{overfullrule}s.
\begin{MacroCode}{class}
\end{MacroCode}

    Load \file{leqno.clo} and \file{fleqno.clo} if those options are desired.
\begin{MacroCode}{class}
\end{MacroCode}

    \begin{macro*}{\@titstyle}
    \changes{0.11a}{Removed incorrect \cs{bfseries}, replaced \cs{relax} with \cs{rmfamily}}
    Set the title font as prescribed.
\begin{MacroCode}{class}

\end{MacroCode}
    \end{macro*}

    Now things get tricky. If we are using \hologo{XeTeX}, we don't
    want \pkg{fontenc} because it makes no sense. We \emph{do} want
    \pkg{fontspec}, though, because we want to be able to specify
    fonts. We also include a \cs{nobreakspace} macro which seems to
    be broken in \hologo{XeTeX}.
\begin{MacroCode}{class}

\end{MacroCode}

    The \opt{color} option requires both \pkg{etoolbox} and
    \pkg{xcolor} to be loaded.
\begin{MacroCode}{class}
\ifskrapport@color
\fi
\end{MacroCode}

    We also need to handle the \opt{font} option. Note that we only
    load \pkg{tpagella} if it exists, and fall back to \pkg{mathpazo}
    otherwise.
\begin{MacroCode}{class}    

\end{MacroCode}

    Language options follow. It it a bit convoluted since we load
    \pkg{polyglossia} in \hologo{XeTeX} and \pkg{babel} in \TeX,
    but it's fairly easy to follow.
\begin{MacroCode}{class}

\end{MacroCode}

    Two-column mode requires \pkg{etoolbox} and \pkg{multicol}.
\begin{MacroCode}{class}
\ifskrapport@twocolumn
\fi
\end{MacroCode}

    \subsubsection{Required packages}
    We also include some essential packages per default. The
    \pkg{calc} package, for instance, is essential in later
    definitions.
\begin{MacroCode}{class}
\RequirePackage{calc}
\end{MacroCode}

    At the end of the class definition we load a couple of very
    useful packages that improve typesetting. These are
    \pkg{microtype}, \pkg{icomma} and \pkg{amsmath}. Additionally,
    we load \pkg{skmath} if it exists and is wanted, and unless we
    are using KP-Fonts, we load \pkg{amssymb} and \pkg{textcomp} (the
    reason being that \pkg{kpfonts} load these packages itself, see
    \textcite[p.~1]{kpfonts}, and we don't want conflicting options).
\begin{MacroCode}{class}
\AtEndOfClass{
    \RequirePackage{microtype}
    \RequirePackage{icomma}
    \let\Finv\relax
    \let\Game\relax
    \let\beth\relax
    \let\gimel\relax
    \let\daleth\relax
}
\end{MacroCode}
    
    When the document starts, we set the URL style if the user has
    loaded the \pkg{url} package.
\begin{MacroCode}{class}
\AtBeginDocument{
    \@ifundefined{urlstyle}{}{\urlstyle{same}}
}
\end{MacroCode}

    Lastly, we include the correct point size \texttt{.clo} file from the
    \pkg{article} class.
\begin{MacroCode}{class}
\end{MacroCode}
    
    \subsection{Hidden things}
    Set some default measurements.
\begin{MacroCode}{class}
\addtolength\textwidth{0.5\oddsidemargin}
\addtolength\textwidth{0.5\evensidemargin}
\addtolength\oddsidemargin{-0.5\oddsidemargin}
\addtolength\evensidemargin{-0.5\evensidemargin}
\setlength\lineskip{1\p@}
\setlength\normallineskip{1\p@}
\end{MacroCode}
    \begin{macro}{\baselinestretch}
\begin{MacroCode}{class}
\renewcommand\baselinestretch{}
\end{MacroCode}
    \end{macro}

    \subsubsection{Indentation}
    Indentation code. Undo the indentation set by the \pkg{article} 
    class if indentation was requested by the user. See the
    \pkg{parskip} package for further information. Enable french
    spacing as well.
\begin{MacroCode}{class}
\ifskrapport@indent\else
\fi
\frenchspacing
\end{MacroCode}

    \subsubsection{Penalties}
\begin{MacroCode}{class}
\@lowpenalty   51
\@medpenalty  151
\@highpenalty 301
\setcounter{topnumber}{2}
\setcounter{bottomnumber}{1}
\setcounter{totalnumber}{4}
\setcounter{dbltopnumber}{2}
\end{MacroCode}
    \begin{macro*}{\topfraction}
    \begin{macro*}{\bottomfraction}
    \begin{macro*}{\textfraction}
    \begin{macro*}{\floatpagefraction}
    \begin{macro*}{\dbltopfraction}
    \begin{macro*}{\dblfloatpagefraction}
\begin{MacroCode}{class}
\renewcommand\topfraction{.75}
\renewcommand\bottomfraction{.5}
\renewcommand\textfraction{.25}
\renewcommand\floatpagefraction{.625}
\renewcommand\dbltopfraction{.75}
\renewcommand\dblfloatpagefraction{.625}
\end{MacroCode}
    \end{macro*}
    \end{macro*}
    \end{macro*}
    \end{macro*}
    \end{macro*}
    \end{macro*}

    \subsection{User-level commands}
    \subsubsection{The front page}
    \begin{macro}{\author}
    \changes{0.10b}{Improved \cs{author} macro}
    The \cs{author} macro is redefined to accept an optional argument
    and to be used multiple times.
    \begin{macro*}{\skrapport@email}
    The \cs{skrapport@email} helper macro typesets an email address
    using \pkg{hyperref} if that package is used. This is suboptimal,
    the macro behaves differently with respect to special characters
    depending on wether \pkg{hyperref} is loaded or not.
\begin{MacroCode}{class}
\ProvideDocumentCommand\skrapport@email{m}{\texttt{#1}}
\AtBeginDocument{
    \@ifpackageloaded{hyperref}{%
        \DeclareDocumentCommand\skrapport@email{m}{%
            \href{mailto:#1}{\nolinkurl{#1}}%
        }%
    }{}%
}
\end{MacroCode}{class}
    \end{macro*}
    \begin{macro*}{\@author}
\begin{MacroCode}{class}
\def\@author{\relax}
\DeclareDocumentCommand\author{om}{%
    \def\skrapport@tempi{#2\IfNoValueTF{#1}{}{~$\langle$\skrapport@email{#1}$\rangle$}}
    \if\@author\relax
        \xdef\@author{\skrapport@tempi}
    \else
        \expandafter\gdef\expandafter\@author\expandafter{\@author\\[0.5ex]\skrapport@tempi}
    \fi
}
\end{MacroCode}
    \end{macro*}
    \end{macro}

    \begin{macro}{\regarding}
    This macro defines a variable used by \cs{maketitle} to insert a 
    simple text into the header on the title page.
    \begin{macro*}{\@regarding}
\begin{MacroCode}{class}
\def\@regarding{\relax}
\newcommand{\regarding}[1]{\gdef\@regarding{#1}}
\end{MacroCode}
    \end{macro*}
    \end{macro}

    \begin{macro}{\license}
    \changes{0.07}{Added command \cs{license}}
    This macro defines a variable used by \cs{maketitle} to insert a 
    license into the footer on the titlepage.
    \begin{macro*}{\@copyright}
\begin{MacroCode}{class}
\def\@copyright{\relax}
\newcommand{\license}[1]{\gdef\@copyright{#1}}
\end{MacroCode}
    \end{macro*}
    \end{macro}

    \begin{macro}{\maketitle}
    The standard \cs{maketitle} command as taken from the \pkg{article}
    class but with some basic restyling.
\begin{MacroCode}{class}
\let\@smallprintstyle\relax
\newcommand\maketitle{\par%
  \begingroup
    \renewcommand\thefootnote{\@fnsymbol\c@footnote}%
    \def\@makefnmark{\rlap{\@textsuperscript{\normalfont\@thefnmark}}}%
    \long\def\@makefntext##1{\parindent 1em\noindent%
      \hb@xt@1.8em{\hss\@textsuperscript{\normalfont\@thefnmark}}##1}%
    \newpage
    \global\@topnum\z@
    \@maketitle
    \thispagestyle{plain}\@thanks
  \endgroup
  \setcounter{footnote}{0}%
}
\end{MacroCode}
    \end{macro}
    \begin{macro*}{\@maketitle}
\begin{MacroCode}{class}
\def\@maketitle{%
  \newpage
  \null
  \begin{flushleft}%
        \vspace{-\headsep}
    {\small%
      \@smallprintstyle
      \if\@regarding\relax\else\@regarding{, }\fi%
      \@date\par%
    }%
    \vspace{1.5cm}%
    {\Huge\@titstyle\@title\par}%
    \vspace{.125cm}%
    {\Large\@titstyle\@author}%
    \vspace{.75cm}%
  \end{flushleft}%
  \par%
}
\end{MacroCode}
    \end{macro*}

    \begin{environment}{abstract}
    Standard restyled \env{abstract} environment from the
    \pkg{article} class.
\begin{MacroCode}{class}
\newenvironment{abstract}{\newlength\skrapport@abstract@tw\newlength\skrapport@abstract@aw\settowidth{\skrapport@abstract@tw}{\bfseries\abstractname}\setlength{\skrapport@abstract@aw}{\the\textwidth-\the\skrapport@abstract@tw-1em}\begin{minipage}[t]{\skrapport@abstract@tw}\begin{flushright}\leavevmode\bfseries\abstractname\end{flushright}\end{minipage}\hspace{1em}\begin{minipage}[t]{\skrapport@abstract@aw}}{\end{minipage}}
\end{MacroCode}
    \end{environment}

    \begin{environment}{titlepage}
    Titlepage environment.
    \begin{macro*}{\ps@skrapport@titlepage}
\begin{MacroCode}{class}
\newcommand\ps@skrapport@titlepage{\def\@oddhead{}\def\@evenhead{}\def\@oddfoot{\begin{minipage}{\textwidth}\raggedleft\small\par\@smallprintstyle\@copyright\end{minipage}}\let\@evenfoot\@oddfoot}
\end{MacroCode}
    \end{macro*}
\begin{MacroCode}{class}
\newenvironment{titlepage}{\cleardoublepage\setcounter{page}\@ne}{\thispagestyle{skrapport@titlepage}\cleardoublepage\setcounter{page}\@ne}
\end{MacroCode}
    \end{environment}

    \subsubsection{Sectioning}
    Sectioning commands.
\begin{MacroCode}{class}
\setcounter{secnumdepth}{3}
\newcounter{section}
\newcounter{subsection}[section]
\newcounter{subsubsection}[subsection]
\newcounter{paragraph}[subsubsection]
\newcounter{subparagraph}[paragraph]
\end{MacroCode}
    \begin{macro*}{\thesection}
    \begin{macro*}{\thesubsection}
    \begin{macro*}{\thesubsubsection}
    \begin{macro*}{\theparagraph}
    \begin{macro*}{\thesubparagraph}
\begin{MacroCode}{class}
\renewcommand\thesection{\@arabic\c@section}
\renewcommand\thesubsection{\thesection.\@arabic\c@subsection}
\renewcommand\thesubsubsection{\thesubsection.\@arabic\c@subsubsection}
\renewcommand\theparagraph{\thesubsubsection.\@arabic\c@paragraph}
\renewcommand\thesubparagraph{\theparagraph.\@arabic\c@subparagraph}
\end{MacroCode}
    \end{macro*}
    \end{macro*}
    \end{macro*}
    \end{macro*}
    \end{macro*}
    \begin{macro}{\section}
\begin{MacroCode}{class}
\newcommand\section{\@startsection{section}{1}{\z@}%
  {-4ex \@plus 1ex \@minus -1ex}%
  {.5ex \@plus.5ex}%
  {\normalfont\LARGE\@titstyle}}
\end{MacroCode}
    \end{macro}
    \begin{macro}{\subsection}
\begin{MacroCode}{class}
\newcommand\subsection{\@startsection{subsection}{2}{\z@}%
  {-3ex \@plus 1ex \@minus -1ex}%
  {.25ex \@plus.25ex}%
  {\normalfont\Large\@titstyle}}
\end{MacroCode}
    \end{macro}
    \begin{macro}{\subsubsection}
\begin{MacroCode}{class}
\newcommand\subsubsection{\@startsection{subsubsection}{3}{\z@}%
  {-2ex \@plus .5ex \@minus -.5ex}%
  {.125ex \@plus.125ex}%
  {\normalfont\large\@titstyle}}
\end{MacroCode}
    \end{macro}
    \begin{macro}{\paragraph}
\begin{MacroCode}{class}
\newcommand\paragraph{\@startsection{paragraph}{4}{\z@}%
  {1ex \@plus .25ex \@minus -.25ex}%
  {-1em}%
  {\normalfont\normalsize\bfseries}}
\end{MacroCode}
    \end{macro}
    \begin{macro}{\subparagraph}
\begin{MacroCode}{class}
\newcommand\subparagraph{\@startsection{subparagraph}{5}{\parindent}%
  {1ex \@plus .25ex \@minus -.25ex}%
  {-1em}%
  {\normalfont\normalsize\itshape}}
\end{MacroCode}
    \end{macro}

    \subsubsection{Commands from \pkg{article}}
    Itemization commands.
\begin{MacroCode}{class}
\setlength\leftmargini{2em}
\leftmargin\leftmargini
\setlength\leftmarginii{2em}
\setlength\leftmarginiii{1.5em}
\setlength\leftmarginiv{1.5em}
\setlength\leftmarginv{1em}
\setlength\leftmarginvi{1em}
\setlength\labelsep{.5em}
\setlength\labelwidth{\leftmargini}
\addtolength\labelwidth{-\labelsep}
\@beginparpenalty -\@lowpenalty
\@endparpenalty   -\@lowpenalty
\@itempenalty     -\@lowpenalty
\end{MacroCode}
    \begin{macro*}{\theenumi}
    \begin{macro*}{\theenumii}
    \begin{macro*}{\theenumiii}
    \begin{macro*}{\theenumiv}
\begin{MacroCode}{class}
\renewcommand\theenumi{\@arabic\c@enumi}
\renewcommand\theenumii{\@alph\c@enumii}
\renewcommand\theenumiii{\@roman\c@enumiii}
\renewcommand\theenumiv{\@Alph\c@enumiv}
\end{MacroCode}
    \end{macro*}
    \end{macro*}
    \end{macro*}
    \end{macro*}
    \begin{macro*}{\labelenumi}
    \begin{macro*}{\labelenumii}
    \begin{macro*}{\labelenumiii}
    \begin{macro*}{\labelenumiv}
\begin{MacroCode}{class}
\newcommand\labelenumi{\theenumi.}
\newcommand\labelenumii{\theenumii)}
\newcommand\labelenumiii{\theenumiii.}
\newcommand\labelenumiv{\theenumiv.}
\end{MacroCode}
    \end{macro*}
    \end{macro*}
    \end{macro*}
    \end{macro*}
    \begin{macro*}{\p@enumii}
    \begin{macro*}{\p@enumiii}
    \begin{macro*}{\p@enumiiv}
\begin{MacroCode}{class}
\renewcommand\p@enumii{\theenumi}
\renewcommand\p@enumiii{\theenumi(\theenumii)}
\renewcommand\p@enumiv{\p@enumiii\theenumiii}
\end{MacroCode}
    \end{macro*}
    \end{macro*}
    \end{macro*}
    \begin{macro*}{\labelitemi}
    \begin{macro*}{\labelitemii}
    \begin{macro*}{\labelitemiii}
    \begin{macro*}{\labelitemiv}
\begin{MacroCode}{class}
\newcommand\labelitemi{\textbullet}
\newcommand\labelitemii{\textopenbullet}
\newcommand\labelitemiii{\normalfont\bfseries\textendash}
\newcommand\labelitemiv{\textrightarrow}
\end{MacroCode}
    \end{macro*}
    \end{macro*}
    \end{macro*}
    \end{macro*}
    \begin{environment}{description}
\begin{MacroCode}{class}
\newenvironment{description}
  {\list{}{\labelwidth\z@\itemindent-\leftmargin
    \let\makelabel\descriptionlabel}}{\endlist}
\newcommand*\descriptionlabel[1]{\hspace\labelsep\normalfont\bfseries #1}
\end{MacroCode}
    \end{environment}

    Quotation environments.
    \begin{environment}{quote}
\begin{MacroCode}{class}
\newenvironment{quote}{\list{}{\rightmargin\leftmargin}\item\relax\itshape}{\endlist}
\end{MacroCode}
    \end{environment}
    \begin{environment}{quotation}
\begin{MacroCode}{class}
\newenvironment{quotation}{\bigskip\begin{quote}}{\end{quote}\bigskip}
\end{MacroCode}
    \end{environment}
    \begin{environment}{verse}
\begin{MacroCode}{class}
\newenvironment{verse}{\begin{quote}}{\end{quote}}
\end{MacroCode}
    \end{environment}

    \begin{macro}{\appendix}
    Appendix macro.
\begin{MacroCode}{class}
\newcommand\appendix{\par\setcounter{section}{0}\setcounter{subsection}{0}\gdef\thesection{\@Alph\c@section}}
\end{MacroCode}
    \end{macro}

    Old font commands.
    \begin{macro}{\rm}
\begin{MacroCode}{class}
\DeclareOldFontCommand{\rm}{\normalfont\rmfamily}{\mathrm}
\end{MacroCode}
    \end{macro}
    \begin{macro}{\sf}
\begin{MacroCode}{class}
\DeclareOldFontCommand{\sf}{\normalfont\sffamily}{\mathsf}
\end{MacroCode}
    \end{macro}
    \begin{macro}{\tt}
\begin{MacroCode}{class}
\DeclareOldFontCommand{\tt}{\normalfont\ttfamily}{\mathtt}
\end{MacroCode}
    \end{macro}
    \begin{macro}{\bf}
\begin{MacroCode}{class}
\DeclareOldFontCommand{\bf}{\normalfont\bfseries}{\mathbf}
\end{MacroCode}
    \end{macro}
    \begin{macro}{\it}
\begin{MacroCode}{class}
\DeclareOldFontCommand{\it}{\normalfont\itshape}{\mathit}
\end{MacroCode}
    \end{macro}
    \begin{macro}{\sl}
\begin{MacroCode}{class}
\DeclareOldFontCommand{\sl}{\normalfont\slshape}{\@nomath\sl}
\end{MacroCode}
    \end{macro}
    \begin{macro}{\sc}
\begin{MacroCode}{class}
\DeclareOldFontCommand{\sc}{\normalfont\scshape}{\@nomath\sc}
\end{MacroCode}
    \end{macro}

    \begin{macro}{\emph}
    Redefining the \cs{emph} style to be bold when nested.
    \begin{macro*}{\em}
\begin{MacroCode}{class}
\let\@emstyle\relax
\DeclareRobustCommand\em{%
    \@nomath\em%
    \ifdim \fontdimen\@ne\font >\z@%
        \itshape\bfseries%
    \else%
        \itshape%
    \fi%
    \@emstyle%
}
\end{MacroCode}
    \end{macro*}
    \end{macro}

    \begin{macro*}{\footnoterule}
    Footnote code.
    \begin{macro*}{\@makefntext}
\begin{MacroCode}{class}
\renewcommand\footnoterule{%
    \kern-3\p@
    \hrule\@width.4\columnwidth
    \kern2.6\p@}
\newcommand\@makefntext[1]{%
    \parindent 1em%
    \noindent
    \hb@xt@1.8em{\hss\@makefnmark}#1}
\end{MacroCode}
    \end{macro*}
    \end{macro*}

    Basic translatable texts.
    \begin{macro}{\contentsname}
\begin{MacroCode}{class}
\newcommand\contentsname{Innehåll}
\end{MacroCode}
    \end{macro}
    \begin{macro}{\refname}
\begin{MacroCode}{class}
\newcommand\refname{Referenser}
\end{MacroCode}
    \end{macro}
    \begin{macro}{\figurename}
\begin{MacroCode}{class}
\newcommand\figurename{Figur}
\end{MacroCode}
    \end{macro}
    \begin{macro}{\tablename}
\begin{MacroCode}{class}
\newcommand\tablename{Tabell}
\end{MacroCode}
    \end{macro}
    \begin{macro}{\appendixname}
\begin{MacroCode}{class}
\newcommand\appendixname{Bilaga}
\end{MacroCode}
    \end{macro}
    \begin{macro}{\abstractname}
\begin{MacroCode}{class}
\newcommand\abstractname{Sammanfattning}
\end{MacroCode}
    \end{macro}
    \begin{macro}{\today}
\begin{MacroCode}{class}
\def\today{\year--\month--\day}
\end{MacroCode}
    \end{macro}

    \subsubsection{Floats}
    Figure and table floats.
\begin{MacroCode}{class}
\setlength\arraycolsep{5\p@}
\setlength\tabcolsep{6\p@}
\setlength\arrayrulewidth{.4\p@}
\setlength\doublerulesep{2\p@}
\setlength\tabbingsep{\labelsep}
\skip\@mpfootins=\skip\footins
\setlength\fboxsep{3\p@}
\setlength\fboxrule{.4\p@}
\end{MacroCode}
    \begin{macro*}{\theequation}
\begin{MacroCode}{class}
\renewcommand\theequation{\@arabic\c@equation}
\end{MacroCode}
    \end{macro*}
\begin{MacroCode}{class}
\newcounter{figure}\renewcommand\thefigure{\@arabic\c@figure}
\def\fps@figure{tb}
\def\ftype@figure{1}
\def\ext@figure{lof}
\def\fnum@figure{\figurename~\thefigure}
\end{MacroCode}
    \begin{environment}{figure}
    \begin{environment*}{figure*}
\begin{MacroCode}{class}
\newenvironment{figure}{\@float{figure}}{\end@float}
\newenvironment{figure*}{\@dblfloat{figure}}{\end@dblfloat}
\end{MacroCode}
    \end{environment*}
    \end{environment}
\begin{MacroCode}{class}
\newcounter{table}\renewcommand\thetable{\@arabic\c@table}
\def\fps@table{tb}
\def\ftype@table{2}
\def\ext@table{lot}
\def\fnum@table{\tablename~\thetable}
\end{MacroCode}
    \begin{environment}{table}
    \begin{environment*}{table*}
\begin{MacroCode}{class}
\newenvironment{table}{\@float{table}}{\end@float}
\newenvironment{table*}{\@dblfloat{table}}{\end@dblfloat}
\end{MacroCode}
    \end{environment*}
    \end{environment}

    Captions.
\begin{MacroCode}{class}
\let\@captionstyle\relax
\newlength\abovecaptionskip\setlength\abovecaptionskip{10\p@}
\newlength\belowcaptionskip\setlength\belowcaptionskip{10\p@}
\end{MacroCode}
    \begin{macro*}{\@makecaption}
    \changes{0.10c}{Fix missing coloring command for some captions}
\begin{MacroCode}{class}
\long\def\@makecaption#1#2{%
    \vskip\abovecaptionskip
    \sbox\@tempboxa{\small{\bfseries\@captionstyle#1:} \itshape#2}%
    \ifdim \wd\@tempboxa >\hsize
        \small{\bfseries\@captionstyle#1:} \itshape#2\par
    \else
        \global \@minipagefalse
        \hb@xt@\hsize{\hfil\box\@tempboxa\hfil}%
    \fi
    \vskip\belowcaptionskip}
\end{MacroCode}
    \end{macro*}

    \subsubsection{Table of contents}
\begin{MacroCode}{class}
\newcommand\@pnumwidth{1.55em}
\newcommand\@tocrmarg{2em}
\newcommand\@dotsep{1.7}
\setcounter{tocdepth}{5}
\end{MacroCode}
    \begin{macro}{\tableofcontents}
\begin{MacroCode}{class}
\newcommand\tableofcontents{%
  \section*{\contentsname
    \@mkboth{\MakeUppercase\contentsname}{\MakeUppercase\contentsname}
  }%
    \vskip\baselineskip%
  \@starttoc{toc}%
}
\end{MacroCode}
    \end{macro}
    \begin{macro*}{\l@section}
    \begin{macro*}{\l@subsection}
    \begin{macro*}{\l@subsubsection}
    \begin{macro*}{\l@paragraph}
    \begin{macro*}{\l@subparagraph}
\begin{MacroCode}{class}
\newcommand*\l@section{\@dottedtocline{1}{0em}{1.3em}}
\newcommand*\l@subsection{\@dottedtocline{2}{1.3em}{2em}}
\newcommand*\l@subsubsection{\@dottedtocline{3}{3.3em}{3.15em}}
\newcommand*\l@paragraph{\@dottedtocline{4}{6.45em}{4.15em}}
\newcommand*\l@subparagraph{\@dottedtocline{5}{10.6em}{5.15em}}
\end{MacroCode}
    \end{macro*}
    \end{macro*}
    \end{macro*}
    \end{macro*}
    \end{macro*}

    Fix for \emph{e.g.}~\pkg{tocloft} package.
\begin{MacroCode}{class}
\let\l@figure\@empty
\let\l@table\@empty
\end{MacroCode}

    \subsubsection{Basic bibliography support}
\begin{MacroCode}{class}
\newdimen\bibindent
\setlength\bibindent{2em}
\end{MacroCode}
    \begin{environment}{thebibliography}
\begin{MacroCode}{class}
\newenvironment{thebibliography}[1]
    {\section*{\refname}%
        \@mkboth{\MakeUppercase\refname}{\MakeUppercase\refname}%
        \list{\@biblabel{\@arabic\c@enumiv}}%
            {\settowidth\labelwidth{\@biblabel{#1}}%
                \leftmargin\labelwidth
                \advance\leftmargin\labelsep
                \@openbib@code
                \usecounter{enumiv}%
                \let\p@enumiv\@empty
                \renewcommand\theenumiv{\@arabic\c@enumiv}}%
        \sloppy
        \clubpenalty4000
        \@clubpenalty \clubpenalty
        \widowpenalty4000}%
    {\def\@noitemerr
        {\@latex@warning{Empty ‘thebibliography’ environment}}%
        \endlist}
\end{MacroCode}
    \end{environment}
    \begin{macro*}{\newblock}
\begin{MacroCode}{class}
\newcommand\newblock{\hskip .11em\@plus.33em\@minus.07em}
\end{MacroCode}
    \end{macro*}
\begin{MacroCode}{class}
\let\@openbib@code\@empty
\end{MacroCode}
    \begin{environment}{theindex}
\begin{MacroCode}{class}
\newenvironment{theindex}
    {\twocolumn[\section*{\indexname}]%
        \@mkboth{\MakeUppercase\indexname}%
            {\MakeUppercase\indexname}%
        \thispagestyle{plain}\parindent\z@
        \parskip\z@ \@plus .3\p@\relax
        \columnseprule \z@
        \columnsep 35\p@
        \let\item\@idxitem}
    {\onecolumn}
\end{MacroCode}
    \end{environment}
    \begin{macro*}{\@idxitem}
\begin{MacroCode}{class}
\newcommand\@idxitem{\par\hangindent 40\p@}
\end{MacroCode}
    \end{macro*}
    \begin{macro}{\subitem}
\begin{MacroCode}{class}
\newcommand\subitem{\@idxitem \hspace*{20\p@}}
\end{MacroCode}
    \end{macro}
    \begin{macro}{\subsubitem}
\begin{MacroCode}{class}
\newcommand\subsubitem{\@idxitem \hspace*{30\p@}}
\end{MacroCode}
    \end{macro}
    \begin{macro*}{\indexspace}
\begin{MacroCode}{class}
\newcommand\indexspace{\par \vskip 10\p@ \@plus5\p@ \@minus3\p@\relax}
\end{MacroCode}
    \end{macro*}

    \subsubsection{Two-column mode}
    The twocolumn hacks implemented.
\begin{MacroCode}{class}
\ifskrapport@twocolumn
    \AtBeginDocument{
\end{MacroCode}
    We include the \pkg{grid} package in two-column mode, because
    it looks way better that way.
\begin{MacroCode}{class}
        \IfStrEq{\skrapport@paper}{a4paper}
            {\def\@@@lines{40}}{\def\@@@lines{28}}
        \IfStrEq{\skrapport@ptsize}{10pt}{
            \RequirePackage[fontsize=10pt,%
                            baseline=12pt,%
                            lines=\@@@lines]{grid}
        }{
            \IfStrEq{\skrapport@ptsize}{11pt}{
                \RequirePackage[fontsize=11pt,%
                                baseline=13.2pt,%
                                lines=\@@@lines]{grid}
            }{
                \RequirePackage[fontsize=12pt,%
                                baseline=14.4pt,%
                                lines=\@@@lines]{grid}
            }   
        }
\end{MacroCode}
    Now, we patch commands. First up is \env{abstract}.
\begin{MacroCode}{class}
        \renewenvironment{abstract}{\section*{\abstractname}}{}
\end{MacroCode}
    Then \env{document} and \cs{maketitle}.
\begin{MacroCode}{class}
        \AfterEndPreamble{\begin{multicols}{2}}
        \AtEndDocument{\end{multicols}}
        \pretocmd{\maketitle}{\end{multicols}}{}{}
        \apptocmd{\maketitle}{\begin{multicols}{2}}{}{}
\end{MacroCode}
    Always typeset \env{equation} and friends outside the grid:
\begin{MacroCode}{class}
        \BeforeBeginEnvironment{equation}{\begin{gridenv}}
        \AfterEndEnvironment{equation}{\end{gridenv}}
        \BeforeBeginEnvironment{equation*}{\begin{gridenv}}
        \AfterEndEnvironment{equation*}{\end{gridenv}}
        \BeforeBeginEnvironment{multline}{\begin{gridenv}}
        \AfterEndEnvironment{multline}{\end{gridenv}}
        \BeforeBeginEnvironment{multline*}{\begin{gridenv}}
        \AfterEndEnvironment{multline*}{\end{gridenv}}
        \BeforeBeginEnvironment{gather}{\begin{gridenv}}
        \AfterEndEnvironment{gather}{\end{gridenv}}
        \BeforeBeginEnvironment{gather*}{\begin{gridenv}}
        \AfterEndEnvironment{gather*}{\end{gridenv}}
        \BeforeBeginEnvironment{align}{\begin{gridenv}}
        \AfterEndEnvironment{align}{\end{gridenv}}
        \BeforeBeginEnvironment{align*}{\begin{gridenv}}
        \AfterEndEnvironment{align*}{\end{gridenv}}
        \BeforeBeginEnvironment{flalign}{\begin{gridenv}}
        \AfterEndEnvironment{flalign}{\end{gridenv}}
        \BeforeBeginEnvironment{flalign*}{\begin{gridenv}}
        \AfterEndEnvironment{flalign*}{\end{gridenv}}
        \BeforeBeginEnvironment{alignat}{\begin{gridenv}}
        \AfterEndEnvironment{alignat}{\end{gridenv}}
        \BeforeBeginEnvironment{alignat*}{\begin{gridenv}}
        \AfterEndEnvironment{alignat*}{\end{gridenv}}
\end{MacroCode}
    The \env{figure} environment is patched...
\begin{MacroCode}{class}
        \expandafter\let\expandafter
                    \old@figurest\csname figure*\endcsname
        \expandafter\let\expandafter
                    \old@endfigurest\csname endfigure*\endcsname
        \RenewDocumentEnvironment{figure}{o}{%
            \begin{gridenv}%
            \vspace{\intextsep}%
            \begin{minipage}{\linewidth}%
            \def\@captype{figure}%
        }{%
            \end{minipage}%
            \vspace{\intextsep}%
            \end{gridenv}%
        }
        \RenewDocumentEnvironment{figure*}{o}{\old@figurest}%
                                             {\old@endfigurest}
\end{MacroCode}
    ...as is \env{table}.
\begin{MacroCode}{class}
        \expandafter\let\expandafter
                    \old@tablest\csname table*\endcsname
        \expandafter\let\expandafter
                    \old@endtablest\csname endtable*\endcsname
        \RenewDocumentEnvironment{table}{o}{%
            \begin{gridenv}%
            \vspace{\intextsep}%
            \begin{minipage}{\linewidth}%
            \def\@captype{table}%
            \let\@old@caption\caption%
            \renewcommand{\caption}[1]{%
                \setlength{\@tempdima}{\abovecaptionskip}%
                \setlength{\abovecaptionskip}{\belowcaptionskip}%
                \setlength{\belowcaptionskip}{\@tempdima}%
                \@old@caption{##1}%
                \vspace{\belowcaptionskip}%
            }%
        }{%
            \end{minipage}%
            \vspace{\intextsep}%
            \end{gridenv}%
        }
        \let\oldoldtablest\oldtablest
        \renewcommand{\oldtablest}{%
            \oldoldtablest%
            \let\@old@caption\caption%
            \renewcommand{\caption}[1]{%
                \setlength{\@tempdima}{\abovecaptionskip}%
                \setlength{\abovecaptionskip}{\belowcaptionskip}%
                \setlength{\belowcaptionskip}{\@tempdima}%
                \@old@caption{##1}%
                \vspace{\belowcaptionskip}%
            }%
        }
        \RenewDocumentEnvironment{table*}{o}{\old@tablest}%
                                            {\old@endtablest}
\end{MacroCode}
    \begin{environment}{onecol}
    \changes{0.10}{Fixed, now not completely broken}
    Finally, we define an environment \env{onecol} that typesets
    arbitrary material in a single column. This is a bit tricky to
    do, and probably cargo-cult as well. We define the start of the
    environment to immediately end itself (with the empty ending),
    then end the \env{multiols} environment, redefine our end macro
    to start \env{multicols} as well as redefining the start of
    \env{onecol} to simply reset itself, then start the environment
    again only to have it ended at once.

    Basically, we trick \LaTeX\ into thinking that we have an empty
    \env{onecol} environment at the end of the first \env{multicols},
    then some content inside a fake \env{onecol}, then an empty
    \env{onecol} at the start of the next \env{multicols}. Voilá, no
    wierd group errors!
\begin{MacroCode}{class}
        \newenvironment{onecol}{
            \end{onecol}
            \end{multicols}
            \begingroup
            \def\endonecol{
                \endgroup
                \begin{multicols}{2}
                \let\old@onecol\onecol
                \def\onecol{
                    \let\onecol\old@onecol
                }
                \begin{onecol}
            }
            \def\onecol{}
            \begin{onecol}
        }{}
    }
\end{MacroCode}
    \end{environment}
\begin{MacroCode}{class}
\fi
\end{MacroCode}

    \subsubsection{Miscellaneous}
    A macro \cs{comment} (alias \cs{com}/\cs{note}) is defined to let 
    the user add comments and notes to the document.
    \begin{macro}{\@comment}
\begin{MacroCode}{class}
\NewDocumentCommand\@comment{m}{%
    {\textbf{Comment:} #1}%
}
\end{MacroCode}
    \end{macro}
    \begin{macro}{\comment}
    \changes{0.10}{Added macro for comments}
\begin{MacroCode}{class}
\NewDocumentCommand\comment{sm}{%
    \IfBooleanTF{#1}%
        {\colorbox{red!50}{\@comment{#2}}}%
        {\marginpar{\@comment{#2}}}%
}
\end{MacroCode}
    \end{macro}
    \begin{macro}{\com}
    \begin{macro}{\note}
    \changes{0.10b}{Fixed a silly error in \cs{com} and \cs{note}}
\begin{MacroCode}{class}
\ProvideDocumentCommand\com{sm}%
    {\IfBooleanTF{#1}{\comment*{#2}}{\comment{#2}}}
\ProvideDocumentCommand\note{sm}%
    {\IfBooleanTF{#1}{\comment*{#2}}{\comment{#2}}}
\end{MacroCode}
    \end{macro}
    \end{macro}

    \subsubsection{Color theme support}
    Color theme setup. Start by patching commands and declaring
    default colors. Not implemented: background colors for e.g.
    quote environments and sections headings, different colors
    for the different sectioning levels.
\begin{MacroCode}{class}
\ifskrapport@color
    \apptocmd{\bfseries}{\color{skrapport@boldcolor}}{}{\ClassError{skrapport}{Could not patch \protect\bfseries}{}}
    \apptocmd{\itshape}{\color{skrapport@italiccolor}}{}{\ClassError{skrapport}{Could not patch \protect\itshape}{}}
    \apptocmd{\@titstyle}{\color{skrapport@titlecolor}}{}{\ClassError{skrapport}{Could not patch \protect\@titstyle}{}}
    \def\@smallprintstyle{\color{skrapport@smallprintcolor}}
    \AtBeginDocument{%
        \let\@abstractname\abstractname
        \def\abstractname{\color{skrapport@titlecolor}\@abstractname}
    }
    \apptocmd{\quote}{\color{skrapport@quotecolor}}{}{}
    \def\@captionstyle{\color{skrapport@captioncolor}}
    \def\@emstyle{\color{skrapport@emphcolor}}
    \ifskrapport@twocolumn\AtBeginDocument{%
        \renewcommand\section{\@startsection {section}{1}{\z@}%
            {-.999\baselineskip}{0.001\baselineskip}{\bfseries\mathversion{bold}\color{skrapport@titlecolor}}}
        \renewcommand\subsection{\@startsection{subsection}{2}{\z@}%
            {\baselineskip}{-.35\baselineskip}{\bfseries\color{skrapport@titlecolor}\unskip}}
        \renewcommand\subsubsection{\@startsection{subsubsection}{3}{\z@}%
            {\baselineskip}{-.35\baselineskip}{\itshape\color{skrapport@titlecolor}\unskip}}
    }\fi
    \renewcommand\paragraph{\@startsection{paragraph}{4}{\z@}%
        {1ex \@plus .25ex \@minus -.25ex}{-1em}{\normalfont\normalsize\bfseries\color{skrapport@titlecolor}}}
    \renewcommand\subparagraph{\@startsection{subparagraph}{5}{\parindent}%
        {1ex \@plus .25ex \@minus -.25ex}{-1em}{\normalfont\normalsize\itshape\color{skrapport@titlecolor}}}
    \AtBeginDocument{\@ifpackageloaded{hyperref}{%
        \hypersetup{%
            citebordercolor=skrapport@citecolor,citecolor=skrapport@citecolor,%
            filebordercolor=skrapport@filecolor,filecolor=skrapport@filecolor,%
            linkbordercolor=skrapport@linkcolor,linkcolor=skrapport@linkcolor,%
            menubordercolor=skrapport@menucolor,menucolor=skrapport@menucolor,%
            urlbordercolor=skrapport@urlcolor,urlcolor=skrapport@urlcolor,%
            runbordercolor=skrapport@runcolor,runcolor=skrapport@runcolor%
        }
    }{}}
    \AtBeginDocument{\color{skrapport@defaultcolor}}
\end{MacroCode}
    \begin{macro}{\colortheme}
    The \cs{colortheme} macro allows the end-user to load color themes
    (described later) to customize the colors of the document when the
    class is loaded with the \opt{color} option.
\begin{MacroCode}{class}
    \newcommand\colortheme[1]{\usepackage{skrapport-colortheme-#1}}
\end{MacroCode}
    \end{macro}
\begin{MacroCode}{class}
    \colortheme{default}
\fi
\end{MacroCode}

    \subsection{Final class setup}
    We end the document class by setting a few lengths along with the
    page style and page numbering. Also, activate \cs{raggedbottom} and
    \cs{onexolumn} (since we always do all the two-column stuff
    ourselves anyway).
\begin{MacroCode}{class}
\setlength\columnsep{10\p@}
\setlength\columnseprule{0\p@}
\pagestyle{plain}
\pagenumbering{arabic}
\raggedbottom
\onecolumn
\endinput
\end{MacroCode}

    \subsection{Color themes}
    As described earlier, the user can load color themes to customize
    the appearance of the document if the class was loaded with the
    \opt{color} option. Four themes are available by default.
    
    \subsubsection{Default color theme}
    \begin{theme}{Default}
    \changes{0.09}{Added default color theme}
\begin{MacroCode}{theme-default}
\NeedsTeXFormat{LaTeX2e}[1999/12/01]
\ProvidesPackage{skrapport-colortheme-default}%
    [2012/06/07 v1.0 skrapport color theme default]
\definecolor{skrapport@citecolor}{named}{green}
\definecolor{skrapport@filecolor}{named}{teal}
\definecolor{skrapport@linkcolor}{named}{red}
\definecolor{skrapport@menucolor}{named}{red}
\definecolor{skrapport@urlcolor}{named}{cyan}
\definecolor{skrapport@runcolor}{named}{teal}
\definecolor{skrapport@boldcolor}{named}{black}
\definecolor{skrapport@titlecolor}{named}{black}
\definecolor{skrapport@italiccolor}{named}{black}
\definecolor{skrapport@smallprintcolor}{named}{black}
\definecolor{skrapport@quotecolor}{named}{black}
\definecolor{skrapport@captioncolor}{named}{black}
\definecolor{skrapport@emphcolor}{named}{black}
\definecolor{skrapport@defaultcolor}{named}{black}
\endinput
\end{MacroCode}
    \end{theme}

    \subsubsection{Unscathed color theme}
    \begin{theme}{Unscathed}
    \changes{0.09}{Added ``Unschathed'' color theme}
\begin{MacroCode}{theme-unscathed}
\NeedsTeXFormat{LaTeX2e}[1999/12/01]
\ProvidesPackage{skrapport-colortheme-unscathed}%
    [2012/06/07 v1.0 skrapport color theme unscathed]
\definecolor{skrapport@citecolor}{named}{skrapport@defaultcolor}
\definecolor{skrapport@filecolor}{named}{skrapport@defaultcolor}
\definecolor{skrapport@menucolor}{named}{skrapport@defaultcolor}
\definecolor{skrapport@runcolor}{named}{skrapport@defaultcolor}
\definecolor{skrapport@emphcolor}{HTML}{463335}
\definecolor{skrapport@linkcolor}{HTML}{CF5D3B}
\definecolor{skrapport@urlcolor}{named}{skrapport@linkcolor}
\definecolor{skrapport@titlecolor}{HTML}{B34430}
\definecolor{skrapport@captioncolor}{named}{skrapport@titlecolor}
\definecolor{skrapport@quotecolor}{HTML}{70524A}
\definecolor{skrapport@smallprintcolor}{named}{skrapport@quotecolor}
\endinput
\end{MacroCode}
    \end{theme}

    \subsubsection{Cruelwater color theme}
    \begin{theme}{Cruelwater}
    \changes{0.09}{Added ``Cruelwater'' color theme}
\begin{MacroCode}{theme-cruelwater}
\NeedsTeXFormat{LaTeX2e}[1999/12/01]
\ProvidesPackage{skrapport-colortheme-cruelwater}%
    [2012/06/07 v1.0 skrapport color theme cruelwater]
\definecolor{skrapport@citecolor}{named}{black}
\definecolor{skrapport@filecolor}{named}{black}
\definecolor{skrapport@linkcolor}{named}{black}
\definecolor{skrapport@menucolor}{named}{black}
\definecolor{skrapport@urlcolor}{named}{black}
\definecolor{skrapport@runcolor}{named}{black}
\definecolor{skrapport@boldcolor}{HTML}{030C22}
\definecolor{skrapport@titlecolor}{HTML}{20293F}
\definecolor{skrapport@italiccolor}{named}{black}
\definecolor{skrapport@smallprintcolor}{HTML}{A9B0B3}
\definecolor{skrapport@quotecolor}{HTML}{404749}
\definecolor{skrapport@captioncolor}{HTML}{030C22}
\definecolor{skrapport@emphcolor}{HTML}{20293F}
\definecolor{skrapport@defaultcolor}{named}{black}
\endinput
\end{MacroCode}
    \end{theme}

    \subsubsection{Violet color theme}
    \begin{theme}{Violet}
\changes{0.09}{Added ``Violet'' color theme}
\begin{MacroCode}{theme-violet}
\NeedsTeXFormat{LaTeX2e}[1999/12/01]
\ProvidesPackage{skrapport-colortheme-violet}%
    [2012/06/07 v1.0 skrapport color theme violet]
\definecolor{skrapport@citecolor}{HTML}{932444}
\definecolor{skrapport@filecolor}{HTML}{932444}
\definecolor{skrapport@linkcolor}{HTML}{932444}
\definecolor{skrapport@menucolor}{HTML}{932444}
\definecolor{skrapport@urlcolor}{HTML}{932444}
\definecolor{skrapport@runcolor}{HTML}{932444}
\definecolor{skrapport@boldcolor}{HTML}{311A2A}
\definecolor{skrapport@titlecolor}{HTML}{311A2A}
\definecolor{skrapport@italiccolor}{named}{black}
\definecolor{skrapport@smallprintcolor}{HTML}{D6CBCF}
\definecolor{skrapport@quotecolor}{HTML}{463335}
\definecolor{skrapport@captioncolor}{HTML}{311A2A}
\definecolor{skrapport@emphcolor}{HTML}{98758D}
\definecolor{skrapport@defaultcolor}{named}{black}
\endinput
\end{MacroCode}
    \end{theme}

    \subsubsection{skdoc color theme}
    \begin{theme}{skdoc}
\changes{0.11a}{Added ``skdoc'' color theme}
\begin{MacroCode}{theme-skdoc}
\NeedsTeXFormat{LaTeX2e}[1999/12/01]
\ProvidesPackage{skrapport-colortheme-skdoc}%
    [2013/04/10 v1.0 skrapport color theme skdoc]
\definecolor{skrapport@citecolor}{RGB}{140,35,24}
\definecolor{skrapport@filecolor}{RGB}{73,10,61}
\definecolor{skrapport@linkcolor}{RGB}{140,35,24}
\definecolor{skrapport@menucolor}{RGB}{140,35,24}
\definecolor{skrapport@urlcolor}{RGB}{73,10,61}
\definecolor{skrapport@runcolor}{RGB}{73,10,61}
\definecolor{skrapport@boldcolor}{RGB}{73,10,61}
\definecolor{skrapport@titlecolor}{RGB}{11,72,107}
\definecolor{skrapport@italiccolor}{named}{black}
\definecolor{skrapport@smallprintcolor}{RGB}{11,72,107}
\definecolor{skrapport@quotecolor}{RGB}{140,35,24}
\definecolor{skrapport@captioncolor}{RGB}{11,72,107}
\definecolor{skrapport@emphcolor}{RGB}{73,10,61}
\definecolor{skrapport@defaultcolor}{named}{black}
\endinput
\end{MacroCode}
    \end{theme}

    \Finale
    \section{Installation}
    The easiest way to install this package is using the package
    manager provided by your \LaTeX\ installation if such a program
    is available. Failing that, provided you have obtained the package
    source (\file{skrapport.tex} and \file{Makefile}) from either CTAN
    or Github, running \texttt{make install} inside the source directory
    works well. This will extract the documentation and code from
    \file{skrapport.tex}, install all files into the TDS tree at
    \texttt{TEXMFHOME} and run \texttt{mktexlsr}.

    If you want to extract code and documentation without installing
    the package, run \texttt{make all} instead. If you insist on not
    using \texttt{make}, remember that packages distributed using
    \pkg{skdoc} must be extracted using \texttt{pdflatex}, \emph{not}
    \texttt{tex} or \texttt{latex}.

    \PrintChanges
    \PrintIndex
    \printbibliography
\end{document}
